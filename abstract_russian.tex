\newpage
\pagestyle{plain}

{\selectlanguage{russian}
\begin{center}
    \Large
    \textbf{Аннотация}
\end{center}
В этом тезисе обсуждается геймификация или использование игровых элементов в неигровых контекстах, а также ее потенциальное применение в образовании для усиления мотивации и вовлеченности учащихся. Использование технологий способствует повышению эффективности обучения. Мобильные устройства разрабатываются для того, чтобы они были мощными мини-компьютерами, и потенциал для обучения на этих устройствах становится все больше. Это означает, что люди могут учиться, используя интерактивные технологии, когда и где они пожелают, используя свои собственные персональные устройства. Можно добавить различные элементы геймификации  в различные типы обучения. В дальнейшем можно будет интегрирывать этот метод в государственные учереждения, такие как школы и вузы.
}