\chapter{Introduction}\label{ch:intro}
%these sections are optional, up-to the author
\section{Motivation}
Informally, an algorithm is any well-defined computational procedure that takes some value, or set of values, as input and produces some value, or set of values, as output. An algorithm is thus a sequence of computational steps that transform the input into the output.
We can also view an algorithm as a tool for solving a well-specified computational problem. The statement of the problem specifies in general terms the desired input/output relationship. The algorithm describes a specific computational procedure for achieving that input/output relationship. \\
                    
For example, we might need to sort a sequence of numbers into nondecreasing order. This problem arises frequently in practice and provides fertile ground for introducing many standard design techniques and analysis tools. Here is how we formally define the sorting problem: \\

Input: $\langle a_1, a_2, a_3 ... a_n \rangle$

Output: A permutation (reordering) $\langle a^{'}_1, a^{'}_2, a^{'}_3 ... a^{'}_n \rangle$ \\

For example, given the input sequence $\langle 31, 41, 59, 26, 41, 58 \rangle$ a sorting algorithm returns as output the sequence $\langle 26, 31, 41, 41, 58, 59 \rangle$. Such an input sequence is called an instance of the sorting problem. In general, an instance of a problem consists of the input (satisfying whatever constraints are imposed in the problem statement) needed to compute a solution to the problem. \\

In the present age, people who aspire to work in those fields learn frameworks and libraries that abstract away the use of algorithms and hide the computational cost of each used function. Generally, that is a good thing, because it greatly improves their workflow and makes those professions more accessible, thus reducing the costs of production. However, the downsides are that it affects the software quality, decreasing performance and using large amounts of memory. One of the recent examples of algorithm misuse that appeared globally is a GTA V game, where the if statement was triggered 1.98 billion times while preloading the GTA V online screen.

As you can see, algorithms are the basis of all computational problems, thus it could be seen as an essential tool for everyone in the computer science field including: software engineers, computer science researchers, data scientists, machine learning engineers, data engineers, etc.


\section{Aims and Objectives}
Our project is aimed to make a decent contribution to educating people who aspire to enter career in Computer Science related professions.

\section{Thesis Outline}
The first chapter is an \textit{\nameref{ch:intro}} chapter. This chapter introduces you to gamification and describes aims of this thesis. The second chapter is \textit{\nameref{ch:1}}, which describes the gaming industry. Since it is important to find out why people like to play games, and how to direct this interest in the right direction. Chapter three is the \textit{\nameref{ch:3}}, which explains about implementation experience and basic gamification techniques. In chapter \textit{\nameref{ch:4}} observations and comparisons of gamification material are described.
